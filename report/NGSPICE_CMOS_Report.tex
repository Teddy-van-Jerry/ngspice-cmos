\documentclass{scrartcl}
\usepackage[margin=25mm]{geometry}
\usepackage{newtxtext}
\usepackage{newtxmath}
\usepackage{graphicx}
\usepackage{indentfirst}
\usepackage{svg}
\usepackage{siunitx}
\usepackage{adjustbox}
\usepackage{tikz}
\usepackage{booktabs}
\usepackage{circuitikz}
\usepackage{listings}
\usepackage[style=ieee,backend=biber]{biblatex}
\usepackage[headsepline, autooneside=false, automark]{scrlayer-scrpage}
\usepackage[colorlinks, linkcolor=blue, citecolor=blue, urlcolor=blue]{hyperref}
\clearpairofpagestyles
\ihead{\leftmark}
\ohead{\rightmark}
\cfoot{\thepage}
\usetikzlibrary{calc, positioning}
\ctikzset{
  , bipoles/capacitor/width=.1
  , bipoles/capacitor/height=.3
  , label/align = straight
  , tripoles/pmos style/emptycircle
  , logic ports=ieee
}
\addbibresource{ref.bib}
\definecolor{dkgreen}{rgb}{0,0.5,0}
\definecolor{mauve}{rgb}{0.58,0,0.82}
\makeatletter

% --- a couple of switches to keep track of the context ---
% this switch will be set if we have encountered visible character on the current line
\newif\ifVisCharOccured@spice@
% this switch will be set when we're inside a one-line comment
\newif\ifinlcom@spice@

% flip the switch if visible characters occur on the line
\lst@AddToHook{PostOutput}
{%
  \lst@ifwhitespace%
  \else
    \global\VisCharOccured@spice@true%
  \fi
}

% reset switches at the end of each line
\lst@AddToHook{InitVarsEOL}
{%
  \global\inlcom@spice@false%
  \global\VisCharOccured@spice@false%
}

% reset switches at the beginning of each listing
\lst@AddToHook{PreInit}
{%
  \VisCharOccured@spice@false
  \inlcom@spice@false
}

% helper macro to handle instances of `*'
\newcommand\processlcom@spice
{%
  % if we're already inside a comment, we keep applying the comment style
  \ifinlcom@spice@%
    \lst@commentstyle%
  \else
    % Otherwise, we apply the comment style only if no visible characters have
    % been encountered before the `*' on the current line.
    \ifVisCharOccured@spice@%
    \else
      \global\inlcom@spice@true%
      \lst@commentstyle%
    \fi
  \fi
}
\makeatother
\lstset{
  numbers=left,
  frame=tb,
  aboveskip=3mm,
  belowskip=3mm,
  showstringspaces=false,
  columns=fixed,
  framerule=1pt,
  rulecolor=\color{gray!35},
  backgroundcolor=\color{gray!5},
  basicstyle={\ttfamily\scriptsize},
  numberstyle=\tiny\sffamily\color{gray},
  keywordstyle=[1]\color{blue},
  keywordstyle=[2]\color{brown!80!black},
  commentstyle=\color{dkgreen},
  stringstyle=\color{mauve},
  breaklines=true,
  breakatwhitespace=true,
  tabsize=2,
  extendedchars=false,
  postbreak=\mbox{\hspace{-5.4mm}\textcolor{purple}{$\hookrightarrow$}\space},
  caption={\ttfamily\protect\filename@parse{\lstname}\protect\filename@base\text{.}\protect\filename@ext},
  upquote=true,
  sensitive = false,
  alsoletter = {.},
  comment = *[l][\processlcom@spice]{*},
  string = [b]{'},
  keywords=[1]{.title, .param, .temp, .inc, .ic, .tran, .control, .endc, .end, .ends, .measure, .subckt, .probe},
  keywords=[2]{run, set, hardcopy, if, else, end, shell, print, setcs, let, foreach, alter, alterparam, reset},
}

\newcommand{\includeres}[1]{\sffamily\scriptsize\includesvg[width=\linewidth]{../fig/plot_#1_t.svg}\vspace{-2em}}

\title{NGSPICE Simulation of CMOS Circuits}
\author{Wuqiong Zhao%
  \thanks{\>
    This is the course project of \textsc{Fundamentals of VLSI Desig}n, Southeast University, 2023 Spring.
    This document is available online at \url{https://spice.tvj.one/report/NGSPICE_CMOS_Report.pdf}.\\
    The author is with the National Mobile Communications Research Laboratory,
    Southeast University, Nanjing 211189, China. (E-Mail: \href{mailto:me@wqzhao.org}{me@wqzhao.org})\\
  }}

\begin{document}
  \maketitle

  \begin{abstract}
    NGSPICE is a powerful open-source SPICE simulation software in command line,
    which can efficiently simulate CMOS circuits.
    Basic logic gates, including \textsc{Not}, \textsc{Nand}, \textsc{And}, \textsc{Nor}, are implemented and analyzed.
    The delay parameters and response plots can help understand the circuit characteristics.
    As examples, the 8-input NAND gate with three distinct designs is investigated,
    and the clock controlled SR latch is also simulated to design appropriate MOS parameters.
  \end{abstract}

  \tableofcontents

  % \listoffigures
  % \listoftables

  \section{Environments}

    \subsection{Preparation}
      Install NGSPICE \cite{ngspice} CLI app.
      View the user manual \cite{manual} for more information.

    \subsection{Settings}
      NGSPICE is set to be compatible with HSPICE (see \href{https://github.com/Teddy-van-Jerry/ngspice-cmos/blob/master/.spiceinit}{\texttt{.spiceinit}}).

    \subsection{Development}
      My development environments:
      \begin{itemize}
        \item macOS 13 (Ventura) with M1 chip;
        \item NGSPICE 40 (Homebrew version).
      \end{itemize}
      There is \textit{no} guarantee that the provided code can run on other platforms or other SPICE tools.
      Make changes if appropriate.

  \section{Definitions}
    Delay:
    \begin{itemize}
      \item $t_r$ (\texttt{tr}): rise time (from output crossing $0.1V_{DD}$ to $0.9V_{DD}$);
      \item $t_f$ (\texttt{tf}): fall time (from output crossing $0.9V_{DD}$ to $0.1V_{DD}$);
      \item $t_{pdr}$ (\texttt{tpdf}): rising propagation delay (from input to rising output crossing $V_{DD}/2$);
      \item $t_{pdf}$ (\texttt{tpdf}): falling propagation delay (from input to falling output crossing $V_{DD}/2$);
      \item $t_{pd}$ (\texttt{tpd}): average propagation delay ($t_{pd} = (t_{pdr} + t_{pdf})/2$).
    \end{itemize}

    Operating corner:
    \begin{itemize}
      \item \textbf{SS}: slow-slow;
      \item \textbf{MOM}: nominal (average);
      \item \textbf{FF}: fast-fast.
    \end{itemize}
    

  \section{Sources}

    \subsection{FreePDK45}

      This is a \qty{45}{nm} CMOS library.
      See \href{https://github.com/Teddy-van-Jerry/ngspice-cmos/blob/master/FreePDK45/README}{README} for more information.
      \texttt{TNOM} of FreePDK45 is 27C.

    \subsection{Inverter}\label{s:inv}

      The schematic diagram of the CMOS inverter is shown in Figure~\ref{fig:inv},
      with 1 PMOS and 1 NMOS.
      The design requirement is $t_r=t_f$ when $C_L=\qty{24}{fF}$.
      The designed MOS parameters are shown in Table~\ref{tab:inv}.

      \noindent\begin{minipage}[b]{\linewidth}
        \centering
        \begin{minipage}[b]{.45\linewidth}
          \centering
          \begin{circuitikz}[
  , null n/.style = {
    , inner sep = 0
    , outer sep = 0
    , minimum size = 0
  }
  , la/.style = {
    , font = \sffamily
  }
]
  \draw
    (0, 1) node (M1) [pmos] {M1}
    (0, -1) node (M2) [nmos] {M2}
    (M1.G) to[short, -*] ++ (0, -1) node [null n] (in-branch) {}
    (in-branch) to[short, -o] ++ (-1, 0) node [left, la, mark=*] {in}
    (in-branch) to[short] ++ (0, -1) (M2.G)
    (M1.S) to[short] ++ (0, 0) node [vdd] {$V_{DD}=\qty{1}{V}$}
    (M2.S) to ++ (0, 0) node [tlground] {}
    ($(M1.D)!0.5!(M2.D)$) node [null n] (out-branch) {}
    (M1.D) to (M2.D)
    (out-branch) to[short, *-*] ++ (1, 0) node [right, null n] (L) {}
    (L) to[short, -o] ++ (0.5, 0) node [right, la] {out}
    (L) to[short, C, l={$C_L=\qty{24}{fF}$}] ++ (0, -1) node [tlground] {}
  ;
\end{circuitikz}%

          \captionof{figure}{CMOS inverter schematic.}
          \label{fig:inv}
        \end{minipage}
        \begin{minipage}[b]{.48\linewidth}
          \centering
          \begin{tabular}{ccc}
            \toprule
            MOS & $W$ & $L$ \\\midrule
            PMOS & \qty{360}{nm} & \qty{45}{nm} \\
            NMOS & \qty{225}{nm} & \qty{45}{nm} \\
            \bottomrule
          \end{tabular}
          \captionof{table}{Designed inverter MOS parameters.}
          \label{tab:inv}
        \end{minipage}
      \end{minipage}

      \begin{lstlisting}[title={Inverter Subcircuit}]
.subckt INV gnd i o vdd
  *  src  gate drain body type
  M1 vdd  i    o     vdd  PMOS_VTL W=360nm L=45nm
  M2 gnd  i    o     gnd  NMOS_VTL W=225nm L=45nm
.ends INV
      \end{lstlisting}

      Simulate with \texttt{ngspice inv.cir}, and
      the response of the inverter can be obtained as depicted in Figure~\ref{fig:inv_res}.
      \begin{figure}[htbp]
        \includeres{inv}
        \caption{Response of the inverter.}
        \label{fig:inv_res}
      \end{figure}
      \newpage

    \subsection{NAND2}\label{s:nand2}
      The CMOS NAND2 gate is symmetrically designed with parameters for the worst case.
      The schematic is shown in Figure~\ref{fig:nand2},
      where the value of $W/L$ is labeled next to each MOS.
      The load capacitor is omitted.
      The designed parameters are shown in Table~\ref{tab:nand2}.

      \begin{minipage}[b]{\linewidth}
        \centering
        \begin{minipage}[b]{.45\linewidth}
          \centering
          \begin{circuitikz}[
  , null n/.style = {
    , inner sep = 0
    , outer sep = 0
    , minimum size = 0
  }
  , la/.style = {
    , font = \sffamily
  }
  , P/.style = {
    , pmos
    , font = \footnotesize
  }
  , N/.style = {
    , nmos
    , font = \footnotesize
  }
]
  \draw
    (-1, 1) node (Mp1) [P] {\hspace{-1em}8}
    (1, 1) node (Mp2) [P] {\hspace{-1em}8}
    ([xshift = -7pt]Mp1.S) -- ([xshift=7pt]Mp2.S) node [midway, above] {$V_{DD}$}
    (Mp1.D) -- (Mp2.D)
    (0, -1) node (Mn1) [N] {\hspace{-1em}10}
    (0, -2.5) node (Mn2) [N] {\hspace{-1em}10}
    % (Mn1.S) -- (Mn2.D)
    (Mn1.D) to[short, -*] (Mp1.D -| Mn1.D)
    (Mn1.D) to[short, *-o] ++ (0.5, 0) node [right, la, mark=*] {out}
    (Mn2.S) ++ (0, 0) node [tlground] {}
    (Mp1.G) to[short, -o] ++ (-0, 0) node [left, la, mark=*] {in1}
    (Mp2.G) to[short, -o] ++ (-0, 0) node [left, la, mark=*] {in2}
    (Mn1.G) to[short, -o] ++ (-0, 0) node [left, la, mark=*] {in1}
    (Mn2.G) to[short, -o] ++ (-0, 0) node [left, la, mark=*] {in2}
  ;
\end{circuitikz}%

          \captionof{figure}{CMOS NAND2 schematic.}
          \label{fig:nand2}
        \end{minipage}
        \begin{minipage}[b]{.45\linewidth}
          \centering
          \begin{tabular}{cccc}
            \toprule
            MOS & Number & $W$ & $L$ \\\midrule
            PMOS & 2 & \qty{360}{nm} & \qty{45}{nm} \\
            NMOS & 2 & \qty{225}{nm} & \qty{45}{nm} \\
            \bottomrule
          \end{tabular}
          \captionof{table}{Designed NAND2 MOS parameters.}
          \label{tab:nand2}
        \end{minipage}
      \end{minipage}

      \begin{lstlisting}[title={NAND2 Subcircuit}]
.subckt NAND2 gnd i1 i2 o vdd
  *   src  gate drain body type
  Mp1 vdd  i1   o     vdd  PMOS_VTL W=360nm L=45nm
  Mp2 vdd  i2   o     vdd  PMOS_VTL W=360nm L=45nm
  Mn1 t1   i1   o     gnd  NMOS_VTL W=450nm L=45nm
  Mn2 gnd  i2   t1    gnd  NMOS_VTL W=450nm L=45nm
.ends NAND2
      \end{lstlisting}

      The response simulated with \texttt{ngspice nand2.cir} is shown in Figure~\ref{fig:nand2_res}.
      \begin{figure}[htbp]
        \includeres{nand2}
        \caption{Response of the NAND2 gate.}
        \label{fig:nand2_res}
      \end{figure}

    \vspace*{-.5em}
    \subsection{AND2}

      AND2 is \hyperref[s:nand2]{NAND2} + \hyperref[s:inv]{INV}.
      The response of AND2 at the worst case is shown in Figure~\ref{fig:and2}.
      \vspace*{-.5em}
      \begin{figure}[!h]
        \includeres{and2}
        \caption{Response of the AND2 gate.}
        \label{fig:and2}
        \vspace*{-1em}
      \end{figure}

      \newpage

    \subsection{NOR2}

      The schematic of the NOR2 gate is shown in Figure~\ref{fig:nor2}.
      \begin{figure}[htbp]
        \centering
        \begin{circuitikz}[
  , null n/.style = {
    , inner sep = 0
    , outer sep = 0
    , minimum size = 0
  }
  , la/.style = {
    , font = \sffamily
  }
  , P/.style = {
    , pmos
    , font = \footnotesize
  }
  , N/.style = {
    , nmos
    , font = \footnotesize
  }
]
  \draw
    (0, 1) node (Mp1) [P] {\hspace{-1em}16}
    (0, 2.5) node (Mp2) [P] {\hspace{-1em}16}
    (-1, -1) node (Mn1) [N] {\hspace{-1em}5}
    (1, -1) node (Mn2) [N] {\hspace{-1em}5}
    % (1.4, -1) node [null n] {} % to make sure the above 10 is not cropped
    ([xshift = -7pt]Mn1.S) -- ([xshift=7pt]Mn2.S) node [midway, ground] {}
    (Mn1.D) -- (Mn2.D)
    (Mp1.D) to[short, -*] (Mn1.D -| Mp1.D)
    (Mp1.D) to[short, *-o] ++ (0.5, 0) node [right, la, mark=*] {out}
    (Mp2.S) to[short] ++ (0, 0) node [vdd] {$V_{DD}$}
    (Mp1.G) to[short, -o] ++ (-0, 0) node [left, la, mark=*] {in1}
    (Mp2.G) to[short, -o] ++ (-0, 0) node [left, la, mark=*] {in2}
    (Mn1.G) to[short, -o] ++ (-0, 0) node [left, la, mark=*] {in1}
    (Mn2.G) to[short, -o] ++ (-0, 0) node [left, la, mark=*] {in2}
  ;
\end{circuitikz}%

        \caption{CMOS NOR2 schematic.}
        \label{fig:nor2}
      \end{figure}

      \begin{lstlisting}[title={NOR2 Subcircuit}]
.subckt NOR2 gnd i1 i2 o vdd
  *   src  gate drain body type
  Mp1 t1   i1   o     vdd  PMOS_VTL W=720nm L=45nm
  Mp2 vdd  i2   t1    vdd  PMOS_VTL W=720nm L=45nm
  Mn1 gnd  i1   o     gnd  NMOS_VTL W=225nm L=45nm
  Mn2 gnd  i2   o     gnd  NMOS_VTL W=225nm L=45nm
.ends NOR2
      \end{lstlisting}

      The response of NOR2 gate is given in Figure~\ref{fig:nor2_res}
      when simulating with \texttt{ngspice nor2.cir}.
      \begin{figure}[htbp]
        \includeres{nor2}
        \caption{Response of the AND2 gate.}
        \label{fig:nor2_res}
      \end{figure}

      \newpage
      \subsection{AND8}

      8-input \textsc{And} gate.
      With a large fan-in, there can be several distinct designs.
      Here we want to investigate the performance of different designs.

      The test circuit (defined in \texttt{add8\_test\_inv2.inc})
      involves a \qty{24}{fF} capacitor load at the output,
      and 8 sets of two stages of inverters for each input.
      For the inverter in the test circuit,
      NMOS has $W=\qty{0.75}{\um}$, $L=\qty{0.25}{\um}$,
      and PMOS has $W=\qty{2.60}{\um}$, $L=\qty{0.25}{\um}$.

      The response simulation has the PVT condition of \qty{1.0}{V}, FF, 25°C.

      \subsubsection{Basic Components}

        \paragraph{NAND4A}\label{s:nand4a}
        This design directly extends the structure of NAND2 into NAND4.
        Its schematic is shown in Figure~\ref{fig:nand4}.
        \begin{figure}[htbp]
          \centering
          \begin{circuitikz}[
  , null n/.style = {
    , inner sep = 0
    , outer sep = 0
    , minimum size = 0
  }
  , la/.style = {
    , font = \sffamily
  }
  , P/.style = {
    , pmos
    , font = \footnotesize
  }
  , N/.style = {
    , nmos
    , font = \footnotesize
  }
]
  \draw
    (-3, 1) node (Mp1) [P] {\hspace{-1em}8}
    (-1, 1) node (Mp2) [P] {\hspace{-1em}8}
    (1, 1) node (Mp3) [P] {\hspace{-1em}8}
    (3, 1) node (Mp4) [P] {\hspace{-1em}8}
    ([xshift = -7pt]Mp1.S) -- ([xshift=7pt]Mp4.S) node [midway, above] {$V_{DD}$}
    (Mp1.D) -- (Mp4.D)
    (0, -1) node (Mn1) [N] {\hspace{-1em}20}
    (0, -2.5) node (Mn2) [N] {\hspace{-1em}20}
    (0, -4) node (Mn3) [N] {\hspace{-1em}20}
    (0, -5.5) node (Mn4) [N] {\hspace{-1em}20}
    (Mn1.D) to[short, -*] (Mp1.D -| Mn1.D)
    (Mn1.D) to[short, *-o] ++ (0.5, 0) node [right, la, mark=*] {out}
    (Mn4.S) ++ (0, 0) node [tlground] {}
    (Mp1.G) to[short, -o] ++ (-0, 0) node [left, la, mark=*] {in1}
    (Mp2.G) to[short, -o] ++ (-0, 0) node [left, la, mark=*] {in2}
    (Mp3.G) to[short, -o] ++ (-0, 0) node [left, la, mark=*] {in3}
    (Mp4.G) to[short, -o] ++ (-0, 0) node [left, la, mark=*] {in4}
    (Mn1.G) to[short, -o] ++ (-0, 0) node [left, la, mark=*] {in1}
    (Mn2.G) to[short, -o] ++ (-0, 0) node [left, la, mark=*] {in2}
    (Mn3.G) to[short, -o] ++ (-0, 0) node [left, la, mark=*] {in3}
    (Mn4.G) to[short, -o] ++ (-0, 0) node [left, la, mark=*] {in4}
  ;
\end{circuitikz}%

          \caption{CMOS NAND4A schematic.}
          \label{fig:nand4}
        \end{figure}

        \newpage

        \paragraph{NAND8A}\label{s:nand8a}
        This design directly extends the structure of NAND2 into NAND8.
        Its schematic is shown in Figure~\ref{fig:nand8}.
        \begin{figure}[htbp]
          \centering
          \begin{circuitikz}[
  , null n/.style = {
    , inner sep = 0
    , outer sep = 0
    , minimum size = 0
  }
  , la/.style = {
    , font = \sffamily
  }
  , P/.style = {
    , pmos
    , font = \footnotesize
  }
  , N/.style = {
    , nmos
    , font = \footnotesize
  }
]
  \draw
    (-7, 1) node (Mp1) [P] {\hspace{-1em}8}
    (-5, 1) node (Mp2) [P] {\hspace{-1em}8}
    (-3, 1) node (Mp3) [P] {\hspace{-1em}8}
    (-1, 1) node (Mp4) [P] {\hspace{-1em}8}
    (1, 1) node (Mp5) [P] {\hspace{-1em}8}
    (3, 1) node (Mp6) [P] {\hspace{-1em}8}
    (5, 1) node (Mp7) [P] {\hspace{-1em}8}
    (7, 1) node (Mp8) [P] {\hspace{-1em}8}
    ([xshift = -7pt]Mp1.S) -- ([xshift=7pt]Mp8.S) node [midway, above] {$V_{DD}$}
    (Mp1.D) -- (Mp8.D)
    (0, -1) node (Mn1) [N] {\hspace{-1em}40}
    (0, -2.5) node (Mn2) [N] {\hspace{-1em}40}
    (0, -4) node (Mn3) [N] {\hspace{-1em}40}
    (0, -5.5) node (Mn4) [N] {\hspace{-1em}40}
    (0, -7) node (Mn5) [N] {\hspace{-1em}40}
    (0, -8.5) node (Mn6) [N] {\hspace{-1em}40}
    (0, -10) node (Mn7) [N] {\hspace{-1em}40}
    (0, -11.5) node (Mn8) [N] {\hspace{-1em}40}
    (Mn1.D) to[short, -*] (Mp1.D -| Mn1.D)
    (Mn1.D) to[short, *-o] ++ (0.5, 0) node [right, la, mark=*] {out}
    (Mn8.S) ++ (0, 0) node [tlground] {}
    (Mp1.G) to[short, -o] ++ (-0, 0) node [left, la, mark=*] {in1}
    (Mp2.G) to[short, -o] ++ (-0, 0) node [left, la, mark=*] {in2}
    (Mp3.G) to[short, -o] ++ (-0, 0) node [left, la, mark=*] {in3}
    (Mp4.G) to[short, -o] ++ (-0, 0) node [left, la, mark=*] {in4}
    (Mp5.G) to[short, -o] ++ (-0, 0) node [left, la, mark=*] {in5}
    (Mp6.G) to[short, -o] ++ (-0, 0) node [left, la, mark=*] {in6}
    (Mp7.G) to[short, -o] ++ (-0, 0) node [left, la, mark=*] {in7}
    (Mp8.G) to[short, -o] ++ (-0, 0) node [left, la, mark=*] {in8}
    (Mn1.G) to[short, -o] ++ (-0, 0) node [left, la, mark=*] {in1}
    (Mn2.G) to[short, -o] ++ (-0, 0) node [left, la, mark=*] {in2}
    (Mn3.G) to[short, -o] ++ (-0, 0) node [left, la, mark=*] {in3}
    (Mn4.G) to[short, -o] ++ (-0, 0) node [left, la, mark=*] {in4}
    (Mn5.G) to[short, -o] ++ (-0, 0) node [left, la, mark=*] {in5}
    (Mn6.G) to[short, -o] ++ (-0, 0) node [left, la, mark=*] {in6}
    (Mn7.G) to[short, -o] ++ (-0, 0) node [left, la, mark=*] {in7}
    (Mn8.G) to[short, -o] ++ (-0, 0) node [left, la, mark=*] {in8}
  ;
\end{circuitikz}%

          \caption{CMOS NAND8A schematic.}
          \label{fig:nand8}
        \end{figure}

      \newpage
      \subsubsection{AND8A (Symmetrical Design)}\label{s:and8a}

        This is the most basic case, extending 2-input NAND to 8-input NAND,
        before applying an inverter.
        The schematic of AND8A is shown in Figure~\ref{fig:and8a}.
        MOS parameters are specified in \hyperref[s:nand8a]{NAND8A}.
        \begin{figure}[htbp]
          \centering
          \begin{circuitikz}[
  , null n/.style = {
    , inner sep = 0
    , outer sep = 0
    , minimum size = 0
  }
  , la/.style = {
    , font = \sffamily
  }
  , P/.style = {
    , pmos
    , font = \footnotesize
  }
  , N/.style = {
    , nmos
    , font = \footnotesize
  }
]
  \draw
    (0, 0) node[nand port, number inputs=8] (nand) {}
    (2, 0) node[not port] (inv) {}
    (nand.out) -- (inv.in)
  ;
\end{circuitikz}%

          \caption{CMOS AND8A schematic.}
          \label{fig:and8a}
        \end{figure}

        The simulation result with \texttt{ngspice and8a.cir} is shown in
        Table~\ref{tab:and8a} and Figure~\ref{fig:and8a_res}.
        \begin{table}[htbp]
          \centering
          \caption{AND8A gate simulation result.}
          \label{tab:and8a}
          \begin{tabular}{cS[table-format=3.1]S[table-format=3.1]S[table-format=3.1]S[table-format=3.1]S[table-format=2.3]S[table-format=1.3]}
            \toprule
            PVT Condition   & {$t_r$ (ps)} & {$t_f$ (ps)} & {$t_{pdr}$ (ps)} & {$t_{pdf}$ (ps)} & {P static (\si{\uW})} & {P dynamic (\si{\uW})} \\\midrule
            0.9\,V, SS, 70°C  & 132.6   & 136.9   & 137.3     & 159.9     & 0.076         & 2.366          \\
            1.35\,V, SS, 70°C & 110.6   & 124.4   & 102.4     & 125.9     & 1.023         & 5.591          \\
            1.0\,V, NOM, 25°C & 90.5    & 99.2    & 83.8      & 110.5     & 0.227         & 2.733          \\
            1.5\,V, NOM, 25°C & 79.8    & 95.0    & 69.4      & 92.4      & 6.566         & 9.669          \\
            1.1\,V, FF, 0°C   & 72.5    & 84.6    & 62.7      & 89.4      & 0.955         & 5.321          \\
            1.65\,V, FF, 0°C  & 69.1    & 83.6    & 54.2      & 75.3      & 51.582        & 1.121          \\
          \bottomrule
          \end{tabular}
        \end{table}
        \begin{figure}[htbp]
          \includeres{and8a}
          \caption{Response of the AND8A gate.}
          \label{fig:and8a_res}
        \end{figure}

      \newpage
      \subsubsection{AND8B (NAND4A\texttimes2 + NOR2\texttimes1)}\label{s:and8b}

        The schematic design using 2 NAND4A gates and 1 NOR gate is shown in Figure~\ref{fig:and8b}.
        \begin{figure}[htbp]
          \centering
          \begin{circuitikz}[
  , null n/.style = {
    , inner sep = 0
    , outer sep = 0
    , minimum size = 0
  }
  , la/.style = {
    , font = \sffamily
  }
  , P/.style = {
    , pmos
    , font = \footnotesize
  }
  , N/.style = {
    , nmos
    , font = \footnotesize
  }
]
  \draw
    (0, 1) node[nand port, number inputs=4] (nand1) {}
    (0, -1) node[nand port, number inputs=4] (nand2) {}
    (2.5, 0) node[nor port] (nor) {}
    (nand1.out) -| (nor.in 1)
    (nand2.out) -| (nor.in 2)
  ;
\end{circuitikz}%

          \caption{CMOS AND8B schematic.}
          \label{fig:and8b}
        \end{figure}

        The simulation result with \texttt{ngspice and8b.cir} is shown in
        Table~\ref{tab:and8b} and Figure~\ref{fig:and8b_res}.
        \begin{table}[htbp]
          \centering
          \caption{AND8B gate simulation result.}
          \label{tab:and8b}
          \begin{tabular}{cS[table-format=3.1]S[table-format=3.1]S[table-format=3.1]S[table-format=3.1]S[table-format=2.3]S[table-format=2.3]}
            \toprule
            PVT Condition   & {$t_r$ (ps)} & {$t_f$ (ps)} & {$t_{pdr}$ (ps)} & {$t_{pdf}$ (ps)} & {P static (\si{\uW})} & {P dynamic (\si{\uW})} \\\midrule
            0.9\,V, SS, 70°C  & 118.8   & 131.7   & 102.8     & 106.2     & 0.030         & 2.233           \\
            1.35\,V, SS, 70°C & 96.2    & 113.1   & 78.3      & 83.3      & 0.641         & 6.204           \\
            1.0\,V, NOM, 25°C & 77.2    & 94.9    & 62.7      & 73.0      & 0.125         & 2.592           \\
            1.5\,V, NOM, 25°C & 65.7    & 85.5    & 52.0      & 61.0      & 4.895         & 12.301          \\
            1.1\,V, FF, 0°C   & 59.4    & 79.9    & 46.2      & 58.8      & 0.510         & 5.289           \\
            1.65\,V, FF, 0°C  & 53.3    & 74.3    & 30.5      & 50.3      & 43.767        & 11.248          \\
          \bottomrule
          \end{tabular}
        \end{table}
        \begin{figure}[htbp]
          \includeres{and8b}
          \caption{Response of the AND8B gate.}
          \label{fig:and8b_res}
        \end{figure}

      \newpage
      \subsubsection{AND8C (AND4B\texttimes2 + AND2\texttimes1)}\label{s:and8c}

        An AND4B gate is composed of 2 NAND gates and 1 NOR gate.
        The schematic design of AND8C using 2 NAND4A gates and 1 NOR gate is shown in Figure~\ref{fig:and8c}.
        \begin{figure}[htbp]
          \centering
          \begin{circuitikz}[
  , null n/.style = {
    , inner sep = 0
    , outer sep = 0
    , minimum size = 0
  }
  , la/.style = {
    , font = \sffamily
  }
  , P/.style = {
    , pmos
    , font = \footnotesize
  }
  , N/.style = {
    , nmos
    , font = \footnotesize
  }
]
  \draw
    (0, 0) node[nand port, number inputs=2] (nand) {}
    (2, 0) node[not port] (inv) {}
    (-2.5, 2*0.8) node[nor port] (nor1) {}
    (-2.5, -2*0.8) node[nor port] (nor2) {}
    (-5, 3*0.8) node [nand port] (nand1) {}
    (-5, 1*0.8) node [nand port] (nand2) {}
    (-5, -1*0.8) node [nand port] (nand3) {}
    (-5, -3*0.8) node [nand port] (nand4) {}
    (nor1.out) -| (nand.in 1)
    (nor2.out) -| (nand.in 2)
    (nand1.out) -| (nor1.in 1)
    (nand2.out) -| (nor1.in 2)
    (nand3.out) -| (nor2.in 1)
    (nand4.out) -| (nor2.in 2)
    (nand.out) -- (inv.in)
    (inv.out) to[short, -o] ++ (0, 0) node [right, la] {out}
    (nand1.in 1) to[short, -o] ++ (-0, 0) node [left, la] {in1}
    (nand1.in 2) to[short, -o] ++ (-0, 0) node [left, la] {in2}
    (nand2.in 1) to[short, -o] ++ (-0, 0) node [left, la] {in3}
    (nand2.in 2) to[short, -o] ++ (-0, 0) node [left, la] {in4}
    (nand3.in 1) to[short, -o] ++ (-0, 0) node [left, la] {in5}
    (nand3.in 2) to[short, -o] ++ (-0, 0) node [left, la] {in6}
    (nand4.in 1) to[short, -o] ++ (-0, 0) node [left, la] {in7}
    (nand4.in 2) to[short, -o] ++ (-0, 0) node [left, la] {in8}
  ;
\end{circuitikz}%

          \caption{CMOS AND8C schematic.}
          \label{fig:and8c}
        \end{figure}

        The simulation result with \texttt{ngspice and8c.cir} is shown in
        Table~\ref{tab:and8c} and Figure~\ref{fig:and8c_res}.
        \begin{table}[htbp]
          \centering
          \caption{AND8C gate simulation result.}
          \label{tab:and8c}
          \begin{tabular}{cS[table-format=3.1]S[table-format=3.1]S[table-format=3.1]S[table-format=3.1]S[table-format=2.3]S[table-format=2.3]}
            \toprule
            PVT Condition   & {$t_r$ (ps)} & {$t_f$ (ps)} & {$t_{pdr}$ (ps)} & {$t_{pdf}$ (ps)} & {P static (\si{\uW})} & {P dynamic (\si{\uW})} \\\midrule
            0.9\,V, SS, 70°C  & 118.8   & 120.3   & 103.3     & 105.0     & 0.111         & 3.067          \\
            1.35\,V, SS, 70°C & 96.7    & 102.6   & 79.8      & 83.5      & 1.253         & 9.458          \\
            1.0\,V, NOM, 25°C & 81.5    & 86.9    & 68.4      & 73.3      & 0.301         & 4.164          \\
            1.5\,V, NOM, 25°C & 69.1    & 77.8    & 56.6      & 62.4      & 8.471         & 16.348         \\
            1.1\,V, FF, 0°C   & 65.0    & 73.0    & 53.6      & 60.0      & 1.175         & 6.652          \\
            1.65\,V, FF, 0°C  & 47.9    & 67.8    & 46.9      & 52.2      & 73.357        & 20.798         \\
          \bottomrule
          \end{tabular}
        \end{table}
        \begin{figure}[htbp]
          \includeres{and8c}
          \caption{Response of the AND8C gate.}
          \label{fig:and8c_res}
        \end{figure}

      \newpage
      \subsubsection{Analysis}

        Among the three designs (\hyperref[s:and8a]{AND8A}, \hyperref[s:and8b]{AND8B} and \hyperref[s:and8c]{AND8C}),
        \hyperref[s:and8b]{AND8B} has the smallest latency,
        closely followed by \hyperref[s:and8c]{AND8C},
        and the largest latency is observed with \hyperref[s:and8a]{AND8A}.
        Compared with \hyperref[s:and8b]{AND8B}, the fan-in of \hyperref[s:and8b]{AND8B} is significantly reduced,
        resulting in lower latency.
        Though \hyperref[s:and8c]{AND8C} has an even smaller fan-in, the number of stages in the circuit is larger than that of \hyperref[s:and8b]{AND8B}.
        Thus, \hyperref[s:and8b]{AND8B} achieves a reasonable tradeoff, and has the lowest latency.

        There are also some other observations:
        \begin{itemize}
          \item A larger VDD can reduce the latency to some extent, but resulting in much larger power consumption.
          \item A faster operating corner ($\text{FF} > \text{NOM} > \text{SS}$) will help cut down latency, but also increases power.
          \item A higher temperature increases power in return for a reduced latency.
        \end{itemize}

      \subsection{Clock Controlled SR Latch}

        \subsubsection{Schematic Design}

          The schematic design of the clock controlled SR latch is shown in Figure~\ref{fig:sr_latch},
          with 2 PMOS and 6 NMOS.
          \begin{figure}[htbp]
            \centering
            \begin{circuitikz}[
  , null n/.style = {
    , inner sep = 0
    , outer sep = 0
    , minimum size = 0
  }
  , la/.style = {
    , font = \sffamily
  }
  , P/.style = {
    , pmos
  }
  , N/.style = {
    , nmos
  }
]
  \draw
    (-1.2, -1) node (M1) [N, xscale=-1, label={[xshift=.5em]center:\footnotesize$M_1$}] {}
    (1.2, -1) node (M3) [N, label={[xshift=-.5em]center:\footnotesize$M_3$}] {}
    (-1.2, 1) node (M2) [P, xscale=-1, label={[xshift=.5em]center:\footnotesize$M_2$}] {}
    (1.2, 1) node (M4) [P, label={[xshift=-.5em]center:\footnotesize$M_4$}] {}
    (M1.G) -- (M2.G)
    (M3.G) -- (M4.G)
    (M1.D) -- (M2.D)
    (M3.D) -- (M4.D)
    (M1.D) to[short, *-*] (M1.D -| M3.G)
    (M4.D) to[short, *-*] (M4.D -| M2.G)
    (-1.7, -0.9) node (M6) [N, label={[xshift=-.2em]center:\footnotesize$M_6$}, scale=.7] {}
    (1.7, -0.9) node (M8) [N, label={[xshift=.2em]center:\footnotesize$M_8$}, scale=-.7] {}
    (-1.7, -1.8) node (M5) [N, label={[xshift=-.2em]center:\footnotesize$M_5$}, scale=.7] {}
    (1.7, -1.8) node (M7) [N, label={[xshift=.2em]center:\footnotesize$M_7$}, scale=-.7] {}
    ([xshift=-5mm]M5.S) -- ([xshift=5mm]M7.D) node [midway, ground] {}
    ([xshift=-5mm]M2.S) -- ([xshift=5mm]M4.S) node [midway, above] {$V_{DD}$}
    (M1.S) -- (M1.S |- M7.D)
    (M3.S) -- (M3.S |- M5.S)
    (M6.D) |- (M1.D) node [above left] {$\overline{Q}$}
    (M8.S) |- (M4.D) node [above right] {$Q$}
    (M6.G) to[short, -o] ++ (-0, 0) node [la, left] {clk}
    (M8.G) to[short, -o] ++ (0, 0) node [la, right] {clk}
    (M5.G) to[short, -o] ++ (-0, 0) node [la, left] {$S$}
    (M7.G) to[short, -o] ++ (0, 0) node [la, right] {$R$}
  ;
\end{circuitikz}%

            \caption{CMOS clock controlled SR latch schematic.}
            \label{fig:sr_latch}
          \end{figure}

          \begin{lstlisting}[title={Clock Controlled SR Latch Subcircuit}]
* .param WL = 5
.subckt SR_LATCH_CLK gnd s r clk q qn vdd
  *  src  gate drain body type
  M1 qn   q    gnd   gnd  NMOS_VTL W=     90nm L=45nm
  M2 qn   q    vdd   vdd  PMOS_VTL W=    270nm L=45nm
  M3 q    qn   gnd   gnd  NMOS_VTL W=     90nm L=45nm
  M4 q    qn   vdd   vdd  PMOS_VTL W=    270nm L=45nm
  M5 ts   s    gnd   gnd  NMOS_VTL W={WL*45nm} L=45nm
  M6 qn   clk  ts    gnd  NMOS_VTL W={WL*45nm} L=45nm
  M7 tr   r    gnd   gnd  NMOS_VTL W={WL*45nm} L=45nm
  M8 q    clk  tr    gnd  NMOS_VTL W={WL*45nm} L=45nm
.ends SR_LATCH_CLK
          \end{lstlisting}

          You need to specify the parameter \texttt{WL},
          for example \texttt{.param WL = 5}.

        \newpage
        \subsubsection{MOS W/L Design}
          We need to determine the appropriate $W/L$ for $M_5$ to $M_8$.
          Using a sweep, implemented by \texttt{alterparam} within a \texttt{foreach} loop,
          we can obtain Figure~\ref{fig:sr_latch_wl}.
          \begin{figure}[htbp]
            \includeres{sr_latch_wl}
            \caption{Clock controlled SR latch with different $W/L$ values.}
            \label{fig:sr_latch_wl}
          \end{figure}

          Clearly, we need $W/L>4.5$ (at least $4.2$) for the latch to work properly.
          ($M_1$/$M_3$ and $M_2$/$M_4$ have $W/L$ as $2$ and $6$, respectively.)

        \subsubsection{Simulation}

          The response of NOR2 gate is given in Figure~\ref{fig:sr_latch_res}
          when simulating with \texttt{ngspice SR\_latch\_clk.cir}.
          \begin{figure}[htbp]
            \includeres{sr_latch}
            \caption{Response of the clock controlled SR latch when $W/L=6$.}
            \label{fig:sr_latch_res}
          \end{figure}

  \section{License}

    Copyright \copyright{} 2023 Wuqiong Zhao (\texttt{me@wqzhao.org}).
    This project is distributed by an \href{https://github.com/Teddy-van-Jerry/ngspice-cmos/blob/master/LICENSE}{MIT License}.

  \phantomsection
  \addcontentsline{toc}{section}{References}
  \printbibliography

  \newpage
  \appendix
  \section{Resources}

    \subsection{Links}
      \begin{itemize}
        \item GitHub: \href{https://github.com/Teddy-van-Jerry/ngspice-cmos}{Teddy-van-Jerry/ngspice-cmos}
        \item Website: \href{https://spice.tvj.one}{spice.tvj.one}
      \end{itemize}

    \subsection{Schematic Graph}
      All schematic graphs in this document is drawn with Ti\textit{k}Z using \LaTeX{},
      you can find the source code in the GitHub repository.

  \section{Source Code}

    \subsection{Inverter}
      \lstinputlisting{../inv.inc}
      \lstinputlisting{../inv.cir}

    \subsection{NAND2}
      \lstinputlisting{../nand2.inc}
      \lstinputlisting{../nand2.cir}

    \subsection{AND2}
      \lstinputlisting{../and2.inc}
      \lstinputlisting{../and2.cir}

    \subsection{NOR2}
      \lstinputlisting{../nor2.inc}
      \lstinputlisting{../nor2.cir}

    \subsection{NAND4}
      \lstinputlisting{../nand4a.inc}

    \subsection{AND4}
      \lstinputlisting{../and4b.inc}

    \subsection{AND8}
      \subsubsection{Test Circuit}
        \lstinputlisting{../and8_test_inv2.inc}
        \lstinputlisting{../and8_test_pow.inc}
      \subsubsection{AND8A}
        \lstinputlisting{../and8a.inc}
        \lstinputlisting{../and8a.cir}
      \subsubsection{AND8B}
        \lstinputlisting{../and8b.inc}
        \lstinputlisting{../and8b.cir}
      \subsubsection{AND8C}
        \lstinputlisting{../and8c.inc}
        \lstinputlisting{../and8c.cir}

    \subsection{Clock Controlled SR Latch}
      \lstinputlisting{../SR_latch_clk.inc}
      \lstinputlisting{../SR_latch_clk.cir}

    \subsection{MISC}
      \lstinputlisting{../.spiceinit}

\end{document}
