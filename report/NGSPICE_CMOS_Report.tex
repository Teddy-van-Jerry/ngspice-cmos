\documentclass{scrartcl}
\usepackage[margin=25mm]{geometry}
\usepackage{newtxtext}
\usepackage{newtxmath}
\usepackage{graphicx}
\usepackage{indentfirst}
\usepackage{svg}
\usepackage{siunitx}
\usepackage{adjustbox}
\usepackage{tikz}
\usepackage{circuitikz}
\usepackage{listings}
\usepackage[headsepline, autooneside=false, automark]{scrlayer-scrpage}
\usepackage[colorlinks, linkcolor=blue, citecolor=blue, urlcolor=blue]{hyperref}
\clearpairofpagestyles
\ihead{\leftmark}
\ohead{\rightmark}
\cfoot{\thepage}
\usetikzlibrary{calc, positioning}
\ctikzset{
  , bipoles/capacitor/width=.1
  , bipoles/capacitor/height=.3
  , label/align = straight
  , tripoles/pmos style/emptycircle
}
\definecolor{dkgreen}{rgb}{0,0.5,0}
\definecolor{mauve}{rgb}{0.58,0,0.82}
\makeatletter

% --- a couple of switches to keep track of the context ---
% this switch will be set if we have encountered visible character on the current line
\newif\ifVisCharOccured@spice@
% this switch will be set when we're inside a one-line comment
\newif\ifinlcom@spice@

% flip the switch if visible characters occur on the line
\lst@AddToHook{PostOutput}
{%
  \lst@ifwhitespace%
  \else
    \global\VisCharOccured@spice@true%
  \fi
}

% reset switches at the end of each line
\lst@AddToHook{InitVarsEOL}
{%
  \global\inlcom@spice@false%
  \global\VisCharOccured@spice@false%
}

% reset switches at the beginning of each listing
\lst@AddToHook{PreInit}
{%
  \VisCharOccured@spice@false
  \inlcom@spice@false
}

% helper macro to handle instances of `*'
\newcommand\processlcom@spice
{%
  % if we're already inside a comment, we keep applying the comment style
  \ifinlcom@spice@%
    \lst@commentstyle%
  \else
    % Otherwise, we apply the comment style only if no visible characters have
    % been encountered before the `*' on the current line.
    \ifVisCharOccured@spice@%
    \else
      \global\inlcom@spice@true%
      \lst@commentstyle%
    \fi
  \fi
}
\makeatother
\lstset{
  numbers=left,
  frame=tb,
  aboveskip=3mm,
  belowskip=3mm,
  showstringspaces=false,
  columns=fixed,
  framerule=1pt,
  rulecolor=\color{gray!35},
  backgroundcolor=\color{gray!5},
  basicstyle={\ttfamily\scriptsize},
  numberstyle=\tiny\sffamily\color{gray},
  keywordstyle=\color{blue},
  commentstyle=\color{dkgreen},
  stringstyle=\color{mauve},
  breaklines=true,
  breakatwhitespace=true,
  tabsize=2,
  extendedchars=false,
  postbreak=\mbox{\hspace{-5.4mm}\textcolor{purple}{$\hookrightarrow$}\space},
  caption={\ttfamily\protect\filename@parse{\lstname}\protect\filename@base\text{.}\protect\filename@ext},
  % comment=[s]{*},
  sensitive = false,
  alsoletter = {.},
  comment = *[l][\processlcom@spice]{*},
  string = [b]{'},
  morekeywords = {.title, .param, .temp, .inc, .ic, .tran, .control, .endc, .end, .ends, .measure, .subckt},
}


\title{NGSPICE Simulation of CMOS Circuits}
\author{Wuqiong Zhao%
  \thanks{\>
    This is the course project of \textsc{Fundamentals of VLSI Desig}n, Southeast University, 2023 Spring.\\
    The author is with the National Mobile Communications Research Laboratory,
    Southeast University, Nanjing 211189, China. (E-Mail: \href{mailto:me@wqzhao.org}{me@wqzhao.org})\\
  }}

\begin{document}
  \maketitle

  \begin{abstract}
    SPICE simulation of CMOS Circuits using open-source NGSPICE.
  \end{abstract}

  \tableofcontents

  % \listoffigures
  % \listoftables

  \section{Environments}
  \section{Sources}

    \subsection{FreePDK45}

    \subsection{Inverter}

      The schematic diagram of the CMOS inverter is shown in Figure~\ref{fig:inv}.

      \begin{figure}[htbp]
        \centering
        \begin{circuitikz}[
  , null n/.style = {
    , inner sep = 0
    , outer sep = 0
    , minimum size = 0
  }
  , la/.style = {
    , font = \sffamily
  }
]
  \draw
    (0, 1) node (M1) [pmos] {$M_1$}
    (0, -1) node (M2) [nmos] {$M_2$}
    (M1.G) to[short, -*] ++ (0, -1) node [null n] (in-branch) {}
    (in-branch) to[short, -o] ++ (-1, 0) node [left, la, mark=*] {in}
    (in-branch) to[short] ++ (0, -1) (M2.G)
    (M1.S) to[short] ++ (0, 0) node [vdd] {$V_{DD}=\qty{1}{V}$}
    (M2.S) to ++ (0, 0) node [tlground] {}
    ($(M1.D)!0.5!(M2.D)$) node [null n] (out-branch) {}
    (M1.D) to (M2.D)
    (out-branch) to[short, *-*] ++ (1, 0) node [right, null n] (L) {}
    (L) to[short, -o] ++ (0.5, 0) node [right, la] {out}
    (L) to[short, C, l={$C_L=\qty{24}{fF}$}] ++ (0, -1) node [tlground] {}
  ;
\end{circuitikz}%

        \caption{CMOS inverter schematic.}
        \label{fig:inv}
      \end{figure}

      \begin{figure}[htbp]
        \sffamily
        \adjustbox{max width=\linewidth}{\includesvg[width=1.25\linewidth]{../fig/plot_inv_t.svg}}
        \caption{Response of the inverter.}
      \end{figure}

    \subsection{NAND2}

      \begin{figure}[htbp]
        \sffamily
        \adjustbox{max width=\linewidth}{\includesvg[width=1.25\linewidth]{../fig/plot_nand2_t.svg}}
        \caption{Response of the NAND2 gate.}
      \end{figure}

    \subsection{AND2}

      \begin{figure}[htbp]
        \sffamily
        \adjustbox{max width=\linewidth}{\includesvg[width=1.25\linewidth]{../fig/plot_and2_t.svg}}
        \caption{Response of the AND2 gate.}
      \end{figure}

  \appendix

  \section{Links}
    \begin{itemize}
      \item GitHub: \href{https://github.com/Teddy-van-Jerry/ngspice-cmos}{Teddy-van-Jerry/ngspice-cmos}
      \item Website: \href{https://spice.tvj.one}{spice.tvj.one}
    \end{itemize}

  \section{Source Code}

    \subsection{Inverter}
      \lstinputlisting{../inv.inc}
      \lstinputlisting{../inv.cir}

    \subsection{NAND2}
      \lstinputlisting{../nand2.inc}
      \lstinputlisting{../nand2.cir}

    \subsection{AND2}
      \lstinputlisting{../and2.inc}
      \lstinputlisting{../and2.cir}

\end{document}